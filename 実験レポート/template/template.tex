\documentclass[a4paper,11pt]{ltjsarticle}

% 変数定義(編集してください)
% ============================================
% 変数定義ファイル
% ============================================

% --- 余白設定(mm) ---
\providecommand{\margintop}{25}
\providecommand{\marginbottom}{25}
\providecommand{\marginleft}{25}
\providecommand{\marginright}{25}

% --- 文書情報 ---
\providecommand{\reportTitle}{実験レポートタイトル}
\providecommand{\studentName}{氏名}
\providecommand{\studentId}{学籍番号}
\providecommand{\className}{実験科目名}
\providecommand{\experimentDate}{実験日}
\providecommand{\submissionDate}{\today}
\providecommand{\groupNumber}{班番号}
\providecommand{\partnerNames}{共同実験者}

% --- ヘッダー・フッター ---
\providecommand{\headerright}{\reportTitle}
\providecommand{\footercenter}{\thepage}


% プリアンブル読み込み
% ============================================
% レポート用プリアンブル(参照文献あり・biblatex使用)
% ============================================

% --- ページ設定 ---
\usepackage[margin=25mm]{geometry}
\geometry{
  top=\margintop mm,
  bottom=\marginbottom mm,
  left=\marginleft mm,
  right=\marginright mm
}

% --- 日本語対応 ---
\usepackage{luatexja}
\usepackage{luatexja-fontspec}

% --- フォント設定 ---
\usepackage{fontspec}
\setmainfont{Liberation Serif}  % Times New Roman 互換

% --- 参考文献(biblatex + biber)---
\usepackage[
  backend=biber,
  style=numeric,
  sorting=none,
  maxbibnames=99
]{biblatex}
% \addbibresource{references.bib}  % 使用時にコメント解除

% --- 基本パッケージ ---
\usepackage{graphicx}
\usepackage{array}
\usepackage{tabularx}
\usepackage{booktabs}
\usepackage[x11names]{xcolor}

% --- 数式 ---
\usepackage{amsmath}
\usepackage{amssymb}
\usepackage{amsfonts}
\usepackage{mathtools}
\usepackage{siunitx}
\sisetup{per-mode=symbol}

% --- ヘッダー・フッター ---
\usepackage{fancyhdr}
\pagestyle{fancy}
\fancyhf{}
\lhead{\leftmark}
\rhead{\textbf{\headerright}}
\cfoot{\thepage}
\fancypagestyle{plain}{\fancyhf{}}

% --- 目次・索引 ---
\usepackage{imakeidx}
\usepackage{tocloft}
\setcounter{tocdepth}{2}
\renewcommand{\contentsname}{目次}

% --- 日本語ラベル ---
\renewcommand{\abstractname}{概要}
\renewcommand{\figurename}{図}
\renewcommand{\tablename}{表}
\renewcommand{\today}{\number\year 年\number\month 月\number\day 日}

% --- 行間 ---
\renewcommand{\baselinestretch}{1.5}

% --- ハイパーリンク ---
\usepackage[colorlinks=true, linkcolor=blue, citecolor=red, urlcolor=green]{hyperref}
\hypersetup{
  pdfauthor={\studentName},
  pdftitle={\reportTitle}
}

% --- 参照の賢い表示 ---
\usepackage{cleveref}
\crefname{section}{節}{節}
\Crefname{section}{節}{節}
\crefname{figure}{図}{図}
\crefname{table}{表}{表}
\crefname{equation}{式}{式}


% 参考文献ファイル(使用する場合はコメント解除)
% \addbibresource{references.bib}

\begin{document}

% ============================================
% タイトルページ
% ============================================
\begin{titlepage}
  \centering
  \vspace*{2cm}
  {\LARGE \className \par}
  \vspace{1cm}
  {\Huge\bfseries \reportTitle \par}
  \vspace{2cm}

  \begin{tabular}{rl}
    氏名 & \studentName \\
    学籍番号 & \studentId \\
    班 & \groupNumber \\
    共同実験者 & \partnerNames \\
    実験日 & \experimentDate \\
    提出日 & \submissionDate \\
  \end{tabular}

  \vfill
\end{titlepage}

% ============================================
% 目次
% ============================================
\tableofcontents
\newpage

% ============================================
% 本文
% ============================================

\section{目的}

本実験の目的を記述します。

\section{原理}

実験の原理・理論を説明します。

数式の例:
\begin{equation}
  V = IR
  \label{eq:ohm}
\end{equation}

\cref{eq:ohm}はオームの法則です。

\section{実験方法}

\subsection{使用機器}

\begin{table}[htbp]
  \centering
  \caption{使用機器一覧}
  \label{tab:equipment}
  \begin{tabular}{lll}
    \toprule
    機器名 & 型番 & 備考 \\
    \midrule
    オシロスコープ & XXX-000 & \\
    ファンクションジェネレータ & YYY-111 & \\
    \bottomrule
  \end{tabular}
\end{table}

\subsection{実験手順}

\begin{enumerate}
  \item 手順1
  \item 手順2
  \item 手順3
\end{enumerate}

\section{実験結果}

\subsection{測定データ}

\subsection{グラフ}

% 図の挿入例
% \begin{figure}[htbp]
%   \centering
%   \includegraphics[width=0.8\textwidth]{figure.pdf}
%   \caption{実験結果のグラフ}
%   \label{fig:result}
% \end{figure}

\section{考察}

\begin{infobox}[ポイント]
  重要なポイントをボックスで強調できます。
\end{infobox}

\section{結論}

\section{参考文献}

% 参考文献(使用する場合はコメント解除)
% \printbibliography[heading=none]

% ============================================
% 付録(必要に応じて)
% ============================================
% \appendix
% \section{プログラムコード}
%
% \begin{minted}{python}
% import numpy as np
% import matplotlib.pyplot as plt
%
% x = np.linspace(0, 10, 100)
% y = np.sin(x)
% plt.plot(x, y)
% plt.show()
% \end{minted}

\end{document}
