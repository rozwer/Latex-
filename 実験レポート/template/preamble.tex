% ============================================
% 実験レポート用プリアンブル(詳細版・minted対応)
% ============================================

% --- ページ設定 ---
\usepackage[margin=25mm]{geometry}
\geometry{
  top=\margintop mm,
  bottom=\marginbottom mm,
  left=\marginleft mm,
  right=\marginright mm
}

% --- 日本語対応 ---
\usepackage{luatexja}
\usepackage{luatexja-fontspec}

% --- フォント設定 ---
\usepackage{fontspec}
\setmainfont{Liberation Serif}   % Times New Roman 互換
\setsansfont{Liberation Sans}    % Arial 互換
\setmonofont{Liberation Mono}    % Consolas 互換

% --- 参考文献(biblatex + biber)---
\usepackage[
  backend=biber,
  style=numeric,
  sorting=none,
  maxbibnames=99
]{biblatex}
% \addbibresource{references.bib}  % 使用時にコメント解除

% --- 基本パッケージ ---
\usepackage{graphicx}
\usepackage{array}
\usepackage{tabularx}
\usepackage{booktabs}
\usepackage{multirow}
\usepackage{multicol}
\usepackage[x11names]{xcolor}

% --- 数式 ---
\usepackage{amsmath}
\usepackage{amssymb}
\usepackage{amsfonts}
\usepackage{mathtools}
\usepackage{siunitx}
\sisetup{per-mode=symbol}

% --- コードハイライト(minted)---
\usepackage{minted}
\setminted{
  frame=lines,
  framesep=2mm,
  baselinestretch=1.2,
  fontsize=\footnotesize,
  linenos,
  breaklines
}

% --- アルゴリズム ---
\usepackage[ruled,vlined,linesnumbered]{algorithm2e}
\SetKwInput{KwInput}{入力}
\SetKwInput{KwOutput}{出力}

% --- ボックス ---
\usepackage[most]{tcolorbox}
\newtcolorbox{infobox}[1][]{
  colback=blue!5!white,
  colframe=blue!60!black,
  sharp corners,
  boxrule=0.8pt,
  left=6pt, right=6pt, top=6pt, bottom=6pt,
  title=#1
}
\newtcolorbox{warningbox}[1][]{
  colback=yellow!10!white,
  colframe=orange!80!black,
  sharp corners,
  boxrule=0.8pt,
  left=6pt, right=6pt, top=6pt, bottom=6pt,
  title=#1
}

% --- ヘッダー・フッター ---
\usepackage{fancyhdr}
\pagestyle{fancy}
\fancyhf{}
\lhead{\leftmark}
\rhead{\textbf{\headerright}}
\cfoot{\thepage}
\fancypagestyle{plain}{\fancyhf{}}

% --- 目次・索引 ---
\usepackage{imakeidx}
\usepackage{tocloft}
\setcounter{tocdepth}{3}
\renewcommand{\contentsname}{目次}

% --- 日本語ラベル ---
\renewcommand{\abstractname}{概要}
\renewcommand{\figurename}{図}
\renewcommand{\tablename}{表}
\renewcommand{\today}{\number\year 年\number\month 月\number\day 日}

% --- 行間 ---
\renewcommand{\baselinestretch}{1.5}

% --- ハイパーリンク ---
\usepackage[colorlinks=true, linkcolor=blue, citecolor=red, urlcolor=green]{hyperref}
\hypersetup{
  pdfauthor={\studentName},
  pdftitle={\reportTitle}
}

% --- 参照の賢い表示 ---
\usepackage{cleveref}
\crefname{section}{節}{節}
\Crefname{section}{節}{節}
\crefname{figure}{図}{図}
\crefname{table}{表}{表}
\crefname{equation}{式}{式}
\crefname{algorithm}{アルゴリズム}{アルゴリズム}
\crefname{listing}{リスト}{リスト}

% --- カスタム色 ---
\definecolor{codegreen}{rgb}{0,0.6,0}
\definecolor{codegray}{rgb}{0.5,0.5,0.5}
\definecolor{codepurple}{rgb}{0.58,0,0.82}
\definecolor{backcolour}{rgb}{0.95,0.95,0.92}
